\documentclass[12pt]{article}
%\usepackage{palatino}
\usepackage{amsmath,amssymb}
%\usepackage{latexsym}
\textheight 26cm
\textwidth 16cm
\topmargin -2.2cm
\oddsidemargin -.5cm
\evensidemargin 0 cm
%\setlength{\parskip}{12pt}
%\setlength{\parindent}{0pt}
\newcounter{queno}
\newenvironment{quelist}{\begin{list}{{\arabic{queno}.}}{\usecounter{queno}
\setlength{\leftmargin}{7mm} \setlength{\rightmargin}{0mm}
\setlength{\labelwidth}{3mm} \setlength{\labelsep}{4mm}
\setlength{\itemsep}{7mm}\setlength{\parsep}{0mm}}}{\end{list}}

\begin{document}
\thispagestyle{empty}
\hrule
\vspace{.25cm}
\hfill{\emph{\small David Dritschel \quad (dgd@mcs.st-and.ac.uk)}}

\begin{center}
\LARGE Using the {\bf pvs} suite of Fortran90 codes\\[2mm]
\LARGE to study point vortex dynamics on a sphere \\[2mm]
\end{center}

\hrule
\vspace{0.8cm}
\noindent\underline{\it Download and installation}

\medskip
The {\tt pvs} package can be downloaded from \\[2mm]
{\tt www-vortex.mcs.st-and.ac.uk/$\sim$dgd/rio2012/pvs.tar.gz}

\medskip
Once downloaded, uncompress and expand the archive using \\[2mm]
{\tt gunzip pvs.tar}\\
{\tt tar xvf pvs.tar}\\[2mm]
You will then see a directory {\tt pvs/}, which 
contains a subdirectory {\tt src/}, itself containing a 
suite of Fortran90 source codes (ending in {\tt .f90}) and
a makefile (called ``{\tt makefile}'') which sets up a simulation.  There
is also a subdirectory {\tt doc/} which contains this file ({\tt notes.pdf}).

\vspace{0.5cm}
\noindent\underline{\it Compiling and graphics}

\medskip
To set up and run a simulation, you need an operating system which has
the free Fortran compiler {\tt gfortran}.  Also, for displaying graphical
output, it would be handy to install {\tt whirlgif}, which you can download
from the web for free.

\medskip
When starting a new simulation, \underline{first} 
make a new directory {\it in the} {\tt pvs/} {\it directory}, e.g.\\[2mm]
{\tt mkdir case1}\\[2mm]
Then, go into this directory and copy everything from {\tt src/} there, 
e.g.\\[2mm]
{\tt cd case1}\\
{\tt cp ../src/* .}\\[2mm]
Now follow the instructions in the projects available at\\
{\tt www-vortex.mcs.st-and.ac.uk/$\sim$dgd/rio2012/projects} 
to see what parameters you need to change in the file 
{\tt parameters.f90} --- {\it this is the only file you ever need to edit}.
In particular, you probably need to change the number of point vortices,
{\tt n}, but you can also change how frequently you output data through 
{\tt tsave} and the duration of the simulation through {\tt tmax}.  The
final parameter, {\tt ng}, controls the image resolution of the graphical
output (the default value {\tt 512} is normally adequate).

\medskip
Next, compile all codes with a single command,\\[2mm]
{\tt make all clean}\\[2mm]
Set up your initial conditions using either {\tt powhex} or {\tt trihex}, 
e.g.\\[2mm]
{\tt powhex}\\[2mm]
Just input the parameters it asks for.  Finally, start the simulation
by typing\\[2mm]
{\tt pvs > log \&}\\[2mm]
--- the file {\tt log} gives information about the running job.  It
lists time, energy and the number of time steps taken between data saves.
Type {\tt tail -f log} to monitor the output.

\medskip
When a simulation has finished (which can take a while depending on the
number of vortices {\tt n} and the speed of your computer processor), 
you can view the results with the program {\tt image}.  Just type\\[2mm]
{\tt image}

\medskip
The movie produced by {\tt image} can be viewed using a special command
{\tt xvidi} (which might not work on local machines).  Alternatively,
use the command {\tt b2gif}, which will convert the movie to a gif
movie (using the command {\tt whirlgif}).  All of these commands are
provided in \\[2mm]
{\tt www-vortex.mcs.st-and.ac.uk/$\sim$dgd/rio2012/graphics.tar.gz} \\[2mm]
Download this file in the same ``parent''
directory containing {\tt pvs/}, then uncompress and expand it with \\[2mm]
{\tt gunzip graphics.tar}\\
{\tt tar xvf graphics.tar}\\[2mm]
You will see a directory {\tt graphics}.  Go into it 
and determine its full `path', i.e.\\[2mm]
{\tt cd graphics}\\
{\tt pwd}\\[2mm]
This last command will give you the `current working directory'.
Use this for the variable {\tt gradir} in the scripts 
{\tt xvidi} or {\tt b2gif} --- {\it you will need to edit both files}.
You are now ready.  To run {\tt b2gif},
for example from the subdirectory {\tt case1} illustrated above, 
type e.g.\\[2mm]
{\tt ../../graphics/b2gif} \\[2mm]
and follow the instructions.

\medskip
The gif movie can be displayed using the command\\[2mm]
{\tt animate z000-100.gif}\\[2mm]
assuming the gif movie is called {\tt z000-100.gif}, or you
can display it in any web browser.

\end{document}
